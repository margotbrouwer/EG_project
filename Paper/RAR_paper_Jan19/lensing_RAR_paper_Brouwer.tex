% mnras_template.tex
%
% LaTeX template for creating an MNRAS paper
%
% v3.0 released 14 May 2015
% (version numbers match those of mnras.cls)
%
% Copyright (C) Royal Astronomical Society 2015
% Authors:
% Keith T. Smith (Royal Astronomical Society)

% Change log
%
% v3.0 May 2015
%    Renamed to match the new package name
%    Version number matches mnras.cls
%    A few minor tweaks to wording
% v1.0 September 2013
%    Beta testing only - never publicly released
%    First version: a simple (ish) template for creating an MNRAS paper

%%%%%%%%%%%%%%%%%%%%%%%%%%%%%%%%%%%%%%%%%%%%%%%%%%
% Basic setup. Most papers should leave these options alone.
\documentclass[fleqn,usenatbib]{mnras}
%\pdfminorversion=5

% MNRAS is set in Times font. If you don't have this installed (most LaTeX
% installations will be fine) or prefer the old Computer Modern fonts, comment
% out the following line
%\usepackage{newtxtext,newtxmath}
% Depending on your LaTeX fonts installation, you might get better results with one of these:
%\usepackage{mathptmx}
%\usepackage{txfonts}

% Use vector fonts, so it zooms properly in on-screen viewing software
% Don't change these lines unless you know what you are doing
\usepackage[T1]{fontenc}
\usepackage{ae,aecompl}
\usepackage{hyperref}

%%%%% AUTHORS - PLACE YOUR OWN PACKAGES HERE %%%%%

% Only include extra packages if you really need them. Common packages are:
\usepackage{graphicx}	% Including figure files
\usepackage{amsmath}	% Advanced maths commands
\usepackage{amssymb}	% Extra maths symbols
\usepackage{multirow}
\usepackage[usenames,dvipsnames,svgnames,table]{xcolor}
\usepackage{verbatim}

%\hypersetup{draft}

%%%%%%%%%%%%%%%%%%%%%%%%%%%%%%%%%%%%%%%%%%%%%%%%%%

%%%%% AUTHORS - PLACE YOUR OWN COMMANDS HERE %%%%%

% Please keep new commands to a minimum, and use \newcommand not \def to avoid
% overwriting existing commands. Example:
%\newcommand{\pcm}{\,cm$^{-2}$}	% per cm-squared

\newcommand{\homsun}{\,h^{-1} {\rm M_\odot}}
\newcommand{\hmsun}{\,h^{-2} {\rm M_\odot}}
\newcommand{\msun}{{\rm M_\odot}}
\newcommand{\hkpc}{\, h^{-1}{\rm{kpc}} }
\newcommand{\hMpc}{\, h^{-1}{\rm{Mpc}} }
\newcommand{\magn}{\, {\rm mag} }

\newcommand{\hsmsun}{\,h_{70}^{-2} {\rm M_\odot}}
\newcommand{\hskpc}{\, h_{70}^{-1}{\rm{kpc}} }
\newcommand{\hsMpc}{\, h_{70}^{-1}{\rm{Mpc}} }
\newcommand{\hsGpc}{\, h_{70}^{-1}{\rm{Gpc}} }

\newcommand{\lan}{\langle}
\newcommand{\ra}{\rangle}

\newcommand{\lcdm}{{\rm \Lambda CDM}}
\newcommand{\am}{\, {\rm arcmin}}
\newcommand{\as}{\, {\rm arcsec}}
\newcommand*{\mean}[1]{\overline{#1}}
\newcommand*{\E}[1]{\times 10^{#1}}

\newcommand*{\swap}[2]{#2#1}

%%%%%%%%%%%%%%%%%%%%%%%%%%%%%%%%%%%%%%%%%%%%%%%%%%

%%%%%%%%%%%%%%%%%%% TITLE PAGE %%%%%%%%%%%%%%%%%%%

% Title of the paper, and the short title which is used in the headers.
% Keep the title short and informative.
\title[Extending the RAR with KiDS weak lensing]{Extending the Radial Acceleration Relation using Weak Gravitational Lensing with the Kilo-Degree Survey}
% The list of authors, and the short list which is used in the headers.
% If you need two or more lines of authors, add an extra line using \newauthor
\author[M. M. Brouwer et al.]{Margot M. Brouwer$^{1,2}$\thanks{E-mail:brouwer@astro.rug.nl},
	 %Group 1:
	 %Group 2:
	 %Group 3:
	\\
	\\
	% List of institutions
	$^{1}$Kapteyn Astronomical Institute, University of Groningen, PO Box 800, NL-9700 AV Groningen, the Netherlands.\\
	$^{2}$Institute for Theoretical Physics, University of Amsterdam, Science Park 904, 1098 XH Amsterdam, The Netherlands. \\
}

% These dates will be filled out by the publisher
\date{Accepted XXX. Received YYY; in original form ZZZ}

% Enter the current year, for the copyright statements etc.
\pubyear{2019}

% Don't change these lines
\begin{document}
\label{firstpage}
\pagerange{\pageref{firstpage}--\pageref{lastpage}}
\maketitle

% Abstract of the paper
\begin{abstract}
TBW
\end{abstract}


% Select between one and six entries from the list of approved keywords.
% Don't make up new ones.
\begin{keywords}
gravitational lensing: weak -- Surveys -- methods: statistical -- galaxies: haloes -- cosmology: dark matter, theory -- gravitation.
\\
\end{keywords}

%\newpage
\clearpage

%%%%%%%%%%%%%%%%%%%%%%%%%%%%%%%%%%%%%%%%%%%%%%%%%%

%%%%%%%%%%%%%%%%% BODY OF PAPER %%%%%%%%%%%%%%%%%%


\section{Introduction}
\label{sec:introduction}

It has been known for several decades that the outer regions of galaxies rotate faster than would be expected from Newtonian dynamics based on their luminous baryonic mass. This was first discovered by \cite{rubin1983} through measuring galactic rotation curves of optical disks, and by \cite{bosma1981} through measuring hydrogen profiles at larger radii. The excess gravity implied by these measurements have been generally attributed to an invisible substance named Dark Matter (DM), a term coined earlier by \cite{zwicky1937} when he discovered the missing mass problem by measuring the dynamics of galaxies in clusters. Following more recent observations using Weak gravitational Lensing \cite[WL,][]{hoekstra2004,linden2014,mandelbaum2015}, Baryon Acoustic Oscillations \cite[BAO's,][]{eisenstein2005,blake2011} and the Cosmic Microwave background \cite[CMB,][]{spergel2003,planck2015}, cold dark matter\footnote{DM particles that moved at non-relativistic speeds at the time of recombination, as favoured by measurements of the CBM \cite[]{planck2014} and the Lyman-$\alpha$ forest \cite[]{viel2013}.} (CDM) became a key ingredient of the current standard model of cosmology: the $\lcdm$ model. In this paradigm, CDM accounts for $\Omega_{\rm CDM}=0.266$ of the critical density in the Universe, while the cosmological constant $\Lambda$ used to explain the accelerated expansion of the Universe accounts for $\Omega_{\rm \Lambda}=0.685$.

Although the $\lcdm$ model successfully describes the behavior of DM on a wide range of scales, no conclusive evidence for the existence of DM particles has been found so far \cite[despite years of enormous effort; for an overview, see][]{bertone2005,bertone2018}. This still leaves some room for alternative theories of gravity, such as Modified Newtonian Dynamics  \cite[MOND,][]{milgrom1983} and the more recent theory of Emergent Gravity \cite[EG,][]{verlinde2016}. In these theories particle DM does not exist, and all gravity is due to the baryonic matter (or in the case of EG, the interaction between baryonic matter and the entropy associated with dark energy). One of the main properties of these theories, is that the mass discrepancy in galaxies should correlate strongly with their baryonic mass distribution. Such a correlation is indeed observed, first as the baryonic Tully-Fisher relation \cite[BTFR,][]{tully1977} between the total luminosity of a spiral galaxy and its asymptotic rotation velocity, then as a strong correlation between the mass dependency as a function of radius and the enclosed baryonic mass ...cite...
% and has been named the disk-halo conspiracy, or also the 

\begin{comment}
Points:
- This correlation has indeed been found by several groups (Scarpa 2006; Wu and Kroupa 2015), and has been named the disk-halo conspiracy, or also the baryonic Tully-Fisher relation.
- In particular the latest result from McGaugh (2016) and Lelli (2016b) have measured this correlation unprecedented accuracy.
- Using rotation curves from 153 late-type galaxies, measured by the Spitzer etc..., they showed a tight correlation between the observed gravitational acceleration $g_{obs}$, and the acceleration  expected from the baryonic galaxy mass $g_{bar}$. This sparked the interest of scientist working on alternative theories of gravity.
- However, Navarro (2017) used a range of simplifying assumptions based on galaxy observations and LCDM simulations, in order to create an analytical galaxy+halo model. Based on this model, they compute the LCDM predictions for the RAR (or 'mass discrepancy-radial acceleration relation', as they call it), in particular the value of the acceleration $a_0$ where Newtonian gravity breaks down, and the minimum acceleration $a_{min}$ probed by galaxies. Based on their model, they claimed that the MDAR can be explained within the LCDM framework, at the acceleration scales probed by galaxy rotation curves.
- However, since their model relies on the fact that luminous kinematic tracers in galaxies probe only a limited radial range, they predict that "extending observations to radii well beyond the inner halo regions should lead to systematic deviations from the MDAR".
- The goal of this work is to extend observations of the RAR to lower accelerations, which are not measurable using galaxy rotation curves.
- To this end, we use Weak gravitational Lensing (WL): the perturbation of light from background galaxies (sources) by a foreground gravitational potential (lens), as predicted by General Relativity (GR).
- Using the WL method, we can measure the (apparent) mass distribution of galaxies up to a radius that is $\sim100$ times larger than the radius of the galaxy itself, corresponding to an acceleration scale that is 4 orders of magnitude lower than $a_min$.
- By using the unadulterated WL equations to measure the (apparent) density distribution around foreground galaxies, we necessarily assume that the laws of GR hold with respect to the deflection of light by a gravitational potential.
- In EG, this is explained by the fact that... In MOND ...
- In Section ... we will ...
\end{comment}


\section{Data}
\label{sec:data}

\subsection{KiDS source galaxies}
\label{sec:kids}
Write the beginning.
Need to know:
\begin{itemize}
	\item What changes as we go to KiDS-1000 (K1000 paper?).
\end{itemize}

\subsection{GAMA foreground galaxies}
\label{sec:gama}
Write everything.

\subsection{KiDS foreground selection}
\label{sec:gamalike_kids}
Still need to know:
\begin{itemize}
	\item Maciek's GL-KiDS selection criteria for K1000.
	\item Angus' stellar mass method for K1000.
\end{itemize}

\subsection{MICE mock galaxies}
\label{sec:mice_mocks}
Write everything.

\subsection{Bahamas mock galaxies}
\label{sec:bahamas_mocks}
Written by Kyle?

\section{Data analysis}
\label{sec:analysis}

\subsection{Isolated galaxy selection}
\label{sec:isolation}
Write the beginning.
Still need to know:
\begin{itemize}
	\item how to test the isolation criterion.
\end{itemize}

\subsection{Lensing measurement}
\label{sec:lensing}
Write the beginning.
Still need to know:
\begin{itemize}
	\item How (if?) the GGL-pipeline changes with K1000.
\end{itemize}

\subsection{Conversion to radial acceleration}
\label{sec:conversion}
Still need to know: whether we will use the SIS assumption or linear interpolation.
\begin{itemize}
	\item Test both methods using the Bahamas simulation.
\end{itemize}
	
\section{Theoretical predictions}
\label{sec:predictions}

\subsection{Analytical CDM model}
\label{sec:analytical}
Written by Kyle?

\subsection{Modified Newtonian Dynamics}
\label{sec:MOND}
Write everything.

\subsection{Emergent Gravity}
\label{sec:EG}
Write everything.

\section{Results}
\label{sec:results}
Write when the results are ready.
I still need:
\begin{itemize}
	\item The K1000 lensing catalogues with ANNz redshifts and stellar masses.
	\item The results from the Bahamas simulation.
\end{itemize}

\subsection{Isolated galaxies}

\begin{figure}
	\includegraphics[width=1.0\columnwidth]{Figures/RAR_GAMA+MICE_isolated_strong.pdf}
	\caption{TBW}
	%\label{fig:}
\end{figure}

\subsection{Stellar mass bins}

\begin{figure*}
	\includegraphics[width=1.0\textwidth]{Figures/RAR_GAMA+MICE_4-massbins.pdf}
	\caption{TBW}
	%\label{fig:}
\end{figure*}

\begin{figure*}
	\includegraphics[width=1.0\textwidth]{Figures/RAR_GAMA+MICE_4-massbins_isolated_strong.pdf}
	\caption{TBW}
	%\label{fig:}
\end{figure*}

\begin{figure*}
	\includegraphics[width=1.0\textwidth]{Figures/RAR_KiDS+MICE_massbins-8p5_10p5_10p8_11p1_12p0_transverse.pdf}
	\caption{TBW}
	%\label{fig:}
\end{figure*}


\section{Discussion and conclusion}
\label{sec:discon}
Write at the end.

\section*{Acknowledgements}
Write at the end.

\begin{comment}
V. Demchenko acknowledges the Higgs Centre Nimmo Scholarship and the Edinburgh Global Research Scholarship. J. Harnois-D{\'e}raps is supported by the European Commission under a Marie-Sk{\l}odowska-Curie European Fellowship (EU project 656869). M. Bilicki is supported by the Netherlands Organization for Scientific Research, NWO, through grant number 614.001.451. C. Heymans acknowledges support from the European Research Council under grant number 647112. H. Hoekstra acknowledges support from Vici grant 639.043.512, financed by the Netherlands Organization for Scientific Research. K. Kuijken acknowledges support by the Alexander von Humboldt Foundation. H. Hildebrandt is supported by an Emmy Noether grant (No. Hi 1495/2-1) of the Deutsche Forschungsgemeinschaft. P. Schneider is supported by the Deutsche Forschungsgemeinschaft in the framework of the TR33 `The Dark Universe'. E. van Uitert acknowledges support from an STFC Ernest Rutherford Research Grant, grant reference ST/L00285X/1.

Computations for the $N$-body simulations were performed in part on the Orcinus supercomputer at the WestGrid HPC consortium (\url{www.westgrid.ca}), in part on the GPC supercomputer at the SciNet HPC Consortium. SciNet is funded by: the Canada Foundation for Innovation under the auspices of Compute Canada; the Government of Ontario; Ontario Research Fund - Research Excellence; and the University of Toronto.

This research is based on data products from observations made with ESO Telescopes at the La Silla Paranal Observatory under programme IDs 177.A-3016, 177.A-3017 and 177.A-3018, and on data products produced by Target OmegaCEN, INAF-OACN, INAF-OAPD and the KiDS production team,  on behalf of the KiDS consortium. OmegaCEN and the KiDS production team acknowledge support by NOVA and NWO-M grants. Members of INAF-OAPD and INAF-OACN also acknowledge the support from the Department of Physics \& Astronomy of the University of Padova, and of the Department of Physics of Univ. Federico II (Naples).

GAMA is a joint European-Australasian project based around a spectroscopic campaign using the Anglo-Australian Telescope. The GAMA input catalogue is based on data taken from the Sloan Digital Sky Survey and the UKIRT Infrared Deep Sky Survey. Complementary imaging of the GAMA regions is being obtained by a number of independent survey programs including GALEX MIS, VST KiDS, VISTA VIKING, WISE, Herschel-ATLAS, GMRT and ASKAP providing UV to radio coverage. GAMA is funded by the STFC (UK), the ARC (Australia), the AAO, and the participating institutions. The GAMA website is \url{www.gama-survey.org}.

This work has made use of CosmoHub \cite[]{carretero2017}. CosmoHub has been developed by the Port d'Informaci{\'o}n Cient{\'i}fica (PIC), maintained through a collaboration of the Institut de F{\'i}sica d'Altes Energies (IFAE) and the Centro de Investigaciones Energ{\'e}ticas, Medioambientales y Tecnol{\'o}gicas (CIEMAT), and was partially funded by the ``Plan Estatal de Investigaci{\'o}n Cient{\'i}fica y T{\'e}cnica y de Innovaci{\'o}n'' program of the Spanish government.

This work has made use of {\scshape python} (\url{www.python.org}), including the packages {\scshape numpy} (\url{www.numpy.org}) and {\scshape scipy} (\url{www.scipy.org}). Plots have been produced with {\scshape matplotlib} \cite[]{hunter2007matplotlib}. The mock shear profiles from MICE are computed using {\scshape TreeCorr} (\url{https://pypi.python.org/pypi/TreeCorr}).

\emph{Author contributions:} All authors contributed to the development and writing of this paper. The authorship list is given in three groups: the lead authors (M. Brouwer, V. Demchenko, J. Harnois-D{\'e}raps), followed by two alphabetical groups. The first alphabetical group includes those who are key contributors to both the scientific analysis and the data products. The second group covers those who have either made a significant contribution to the data products, or to the scientific analysis.
end{comment}
\end{comment}
%%%%%%%%%%%%%%%%%%%%%%%%%%%%%%%%%%%%%%%%%%%%%%%%%%

%%%%%%%%%%%%%%%%%%%% REFERENCES %%%%%%%%%%%%%%%%%%

% The best way to enter references is to use BibTeX:

\bibliographystyle{mnras}
\bibliography{biblio}


%%%%%%%%%%%%%%%%%%%%%%%%%%%%%%%%%%%%%%%%%%%%%%%%%%

%%%%%%%%%%%%%%%%%%%%%%%%%%%%%%%%%%%%%%%%%%%%%%%%%%


% Don't change these lines
\bsp	% typesetting comment
\label{lastpage}
\end{document}

% End of mnras_template.tex
